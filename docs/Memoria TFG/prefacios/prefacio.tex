\thispagestyle{empty}

\begin{center}
{\large\bfseries MemeHub \\ Edición Online de Imágenes }\\
\end{center}
\begin{center}
Víctor González Argudo\\
\end{center}

%\vspace{0.7cm}

\vspace{0.5cm}
\noindent{\textbf{Palabras clave}: \textit{Javascript, React.js, Editor de Imágenes, Memes, Software Libre}}
\vspace{0.7cm}

\noindent{\textbf{Resumen}\\}
	En la actualidad la mayoría de editores de imágenes online, concretamente los especializados en
	edición de memes no dan una solución suficientemente buena a los usuarios.
	\\\\
	Este proyecto pretende dar una mejor solución a este problema, crear un editor que facilite
	la edición de imágenes empleando tecnologías actuales, con un aspecto simple e intuitivo, un buen 
	rendimiento y entendiendo las necesidades de sus usuarios objetivo.
	\\
	Por ejemplo, haciendo posible la creación, edición y subida del resultado sin necesidad de 
	emplear en ningún momento el almacenamiento local, siguiendo las tendencias actuales de 
	las aplicaciones web (SaaS, Software as a Service) que pretenden llevar a la nube funcionalidades
	que típicamente se ejecutaban de forma local.

\cleardoublepage

\begin{center}
	{\large\bfseries MemeHub \\ Online Image Editing}\\
\end{center}
\begin{center}
	Víctor González Argudo\\
\end{center}
\vspace{0.5cm}
\noindent{\textbf{Keywords}: \textit{Javascript, React.js, Image Editor, Memes, Open Source}}
\vspace{0.7cm}

\noindent{\textbf{Abstract}\\}
	Currently most of the online image editors, being more specific the online meme generators, are
	not as good as they could be. 
	\\\\
	This project, pretends to be a better solution to this problem, creating an image editor that
	eases the creation of memes, making use of new existing technologies, with a good performance
	and understanding correctly the needs of its target users.
	\\
	For example, enabling the creation, edition and upload of the result without the need to use
	the local storage of the PC, following the current tendencies of web applications (Saas, Software as a Service)
	that tries to upload tipical functionalities that were executed locally to the cloud.

\cleardoublepage

\thispagestyle{empty}

\noindent\rule[-1ex]{\textwidth}{2pt}\\[4.5ex]

D. \textbf{Juan Julián Merelo Guervós}, Profesor(a) del departamento de Arquitectura y Tecnología de 
Computadores de la Universidad de Granada.

\vspace{0.5cm}

\textbf{Informo:}

\vspace{0.5cm}

Que el presente trabajo, titulado \textit{\textbf{MemeHub}},
ha sido realizado bajo mi supervisión por \textbf{Víctor González Argudo}, y autorizo la defensa de dicho trabajo ante el tribunal
que corresponda.

\vspace{0.5cm}

Y para que conste, expiden y firman el presente informe en Granada a Junio de 2018.

\vspace{1cm}

\textbf{El/la director(a)/es: }

\vspace{5cm}

\noindent \textbf{Juan Julián Merelo Guervós}

\chapter*{Agradecimientos}

En primer lugar, me gustaría agradecer a mi tutor, JJ, tanto por el apoyo durante la realización
de este proyecto, como por su ayuda y consejos en otros ámbitos relacionados con el mundo de la 
informática durante la carrera.
\\\\
Y a mi familia y amigos por todo apoyo económico y moral, sin los cuales la realización 
de todo este grado no habría sido posible.

