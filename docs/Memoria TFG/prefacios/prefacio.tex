\thispagestyle{empty}

\begin{center}
{\large\bfseries MemeHub \\ Edición Online de Imágenes }\\
\end{center}
\begin{center}
Víctor González Argudo\\
\end{center}

%\vspace{0.7cm}

\vspace{0.5cm}
\noindent{\textbf{Palabras clave}: \textit{Javascript, React.js, Editor de Imágenes, Memes, Software Libre}}
\vspace{0.7cm}

\noindent{\textbf{Resumen}\\}
	En la actualidad los editores de imágenes online, concretamente los especializados en
	edición de memes, 

\cleardoublepage

\begin{center}
	{\large\bfseries MemeHub \\ Online Image Editing}\\
\end{center}
\begin{center}
	Víctor González Argudo\\
\end{center}
\vspace{0.5cm}
\noindent{\textbf{Keywords}: \textit{Javascript, React.js, Image Editor, Memes, Open Source}}
\vspace{0.7cm}

\noindent{\textbf{Abstract}\\}


\cleardoublepage

\thispagestyle{empty}

\noindent\rule[-1ex]{\textwidth}{2pt}\\[4.5ex]

D. \textbf{Tutora/e(s)}, Profesor(a) del departamento de Arquitectura y Tecnología de 
Computadores de la Universidad de Granada.

\vspace{0.5cm}

\textbf{Informo:}

\vspace{0.5cm}

Que el presente trabajo, titulado \textit{\textbf{MemeHub}},
ha sido realizado bajo mi supervisión por \textbf{Víctor González Argudo}, y autorizo la defensa de dicho trabajo ante el tribunal
que corresponda.

\vspace{0.5cm}

Y para que conste, expiden y firman el presente informe en Granada a Junio de 2018.

\vspace{1cm}

\textbf{El/la director(a)/es: }

\vspace{5cm}

\noindent \textbf{Juan Julián Merelo Guervós}

\chapter*{Agradecimientos}

En primer lugar, me gustaría agradecer a mi tutor, JJ, tanto por el apoyo durante la realización
de este proyecto, como por su ayuda en otros ámbitos relacionados con el mundo de la informática.
\\\\


