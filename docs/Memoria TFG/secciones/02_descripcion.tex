\chapter{Descripción del problema}

Este proyecto pretende facilitar en la medida de lo posible la edición de imágenes que requieren
de una edición rápida y ligera, como es principalmente el caso de los memes.
\\\\
Actualmente, la mayoría de usuarios utilizan aplicaciones locales, instaladas en el propio
ordenador, grandes y pesadas para la edición de imágenes que requieren de cambios simples, 
como puede ser: añadir texto, superponer imágenes encima de una base o crear formar geométricas
para resaltar alguna parte de la imagen base.
\\\\
Los programas de edición locales, como Photoshop o MSPaint, son pesados, con largos tiempos de
carga y lo peor, requieren descargar tanto las fuentes como almacenar el resultado de forma local.
\\\\
El objetivo de este proyecto es dar una solución mejor a este problema, siguiendo la tendencia
actual del desarrollo en la actualidad, que son las aplicaciones web (SaaS, Software as a Service), 
evitando tener que efectuar ninguna descarga y dando la posibilidad de cargar, editar y exportar sin 
necesidad de emplear el almacenamiento local.


