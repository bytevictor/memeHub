\chapter{Planificación}

\section{Metodología utilizada}
    Para la realización de este proyecto se ha seguido una metodología ágil o 'Agile', esta metodología
    es especialmente buena para los proyetos que necesitan de mucha flexibilidad, por ende, por 
    la naturaleza de algo como una aplicación de edición de imagen, que es algo que invita fácilmente a 
    añadir nuevas funcionalidades y a cambiar o adaptar antiguas al nuevo contenido que se añada,
    ha sido escogida para el desarrollo de este trabajo.
    \\\\
    Se ha escogido Kanban como metodología ágil para el desarrollo de este proyecto.
    Como marca el desarrollo ágil, se parte de la idea que queremos implementar, un editor
    con las partes y funcionalidades que hemos definido en anteriores capítulos. Y se comienza
    a descomponer en problemas mas pequeños, que a su vez se descomponen en problemas aún mas 
    pequeños hasta llegar a funcionalidades básicas. 
    Estas funcionalidades básicas son las llamadas historias de usuario (HUs) que se redactan 
    desde el punto de vista del usuario que vaya a utilizar esa funcionalidad básica.

    \begin{figure}[!h]
      \centering
      \noindent\makebox[\textwidth]{
        \includegraphics[scale=0.9]{img/ejemplohu.png}}
      \caption{Ejemplo de una historia de usuario empleada en el proyecto}
    \end{figure}
    
    A su vez estas historias de usuario se pueden descomponer en issues que son tareas para
    el programador que no representan una funcionalidad como tal, por ejemplo crear una 
    función que haga un cálculo concreto. De este modo las tareas se van moviendo de fase
    en el Kanban y se van completando, también se pueden añadir nuevas historias de usuario
    mientras va avanzando el desarrollo ya que se basa en ciclos que se repiten.

%- Busqué librerias y cosas mientras seguí aprendiendo React (planificacion)

%- Despues de unos cuantos proyectos de prueba y elegidas las librerias me lancé con el editor
        %La implementación del software se ha dividido en hitos. Estos, han sido definidos en Github
    %y cada uno de ellos contiene un grupo de \textit{issues} que se corresponden con las distintas
    %mejoras que se han ido incorporando al software a lo largo de su desarrollo.

\section{Seguimiento del desarrollo}
Para el seguimiento y gestión del desarrollo se ha utilizado el software de control de versiones
git\cite{git} en la plataforma Github\cite{Github}.
\\
Se ha creado un github project y se ha utilizado una tabla Kanban para gestionar las historias 
de usuario y las issues del proyecto.
\\\\
El Kanban del proyecto puede verse en el siguiente enlace:
\url{https://github.com/bytevictor/memeHub/projects/1}

\begin{figure}[!h]
  \centering
  \noindent\makebox[\textwidth]{
    \includegraphics[scale=0.4]{img/kanban.png}}
  \caption{Tabla Kanban del proyecto de Github de MemeHub.}
\end{figure}