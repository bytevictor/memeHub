\chapter{Implementación}
Para comenzar con la implementación, se partió del código elaborado como demo en la fase 
de planificación.
Está bastante extendido en el desarrollo de React emplear el paquete create-react-app 
\cite{create-react-app}.
\\
Es un paquete creado por Facebook que se instala con el gestor de paquetes de node 
(npm) que crea una aplicación de React vacía con todo lo básico necesario más 
algunos scripts que ayudan en el desarrollo y en el despliegue, como por ejemplo, el 
script react-run, para ejecutar el servidor de desarrollo, que se recompila
automáticamente al detectar cambios y hace el desarrollo mucho más fluido o el build
para crear una versión release optimizada.
\\\\
Por sí sólo, React es una biblioteca para construir interfaces de usuario que cuenta con
módulos extra que permiten extender su funcionalidad, es decir, no es un framework, como
otras tecnologías empleadas para el diseño de interfaces, como Vue.js o Angular.js, 
sin embargo, con estos módulos podemos ampliar las funcionalidades de la biblioteca React 
dependiendo del uso que vayamos a darle, en nuestro caso como queremos crear una aplicación
web, necesitamos interactuar con el 'Document Object Model' (DOM \cite{DOM}), que es la estructura
de documentos HTML que genera React y procesa el navegador, por ende y tras la creación de 
la plantilla inicial en React, instalamos los módulos de React que permiten interactuar 
con él, react-dom \cite{react-dom}.
\\\\
También necesitamos un módulo que permita a React cambiar la página generada dependiendo de la 
ruta a la que accede el usuario, en el caso de nuestra aplicación, siempre querremos mostrar 
la página del editor así que debemos redirigir todas las rutas al componente principal <Editor/>
que veremos en profundidad más adelante (aunque se planea añadir más páginas en un futuro).
Para ello, empleamos el módulo 'react-router-dom' \cite{react-router-dom} que es el módulo
más utilizado en la comunidad de React para implementar esta funcionalidad.
\\\\
En un primer lugar se pensó introducir una página de error, pero ya que en principio solo hay una
página en toda la web, se han redirigido todas las rutas a la misma, el componente principal.

\begin{lstlisting}[caption={Componente Router de App.js}]
  <Router>
    <Switch>
      <Route exact path="/" component={Editor} />
      <Route exact path="/editor" component={Editor} />
      <Route component={Editor} />
    </Switch>
  </Router>
\end{lstlisting}

Aunque bastaría con emplear una única ruta se han especificado también las rutas '/' y '/editor' 
ya que, como se ha comentado antes, se planea extender la cantidad de páginas en un futuro.

\section{Componente Principal: Editor}

En el desarrollo de React, todo está formado por componentes, como hemos comentado anteriormente,
cada página es un componente, en nuestro caso solo tenemos una página y por ende un gran componente,
el componente Editor.
\\
Este es el componente que renderiza todos los demás subcomponentes que componen la aplicación
y que se encarga de comunicar todas las partes de la misma actuando como un hub.
\\
Como se planificó en el diseño y en la demo, la aplicación cuenta de 3 partes principales.

\begin{figure}[!h]
  \centering
  \includegraphics[scale=0.30]{img/ESQUEMA_PARTES.png}
  \caption{Esquema de las principales partes de la aplicación}
\end{figure}

\newpage
La principal ventaja que tiene dividir la aplicación en estos subcomponentes es que desde
React, podemos re-renderizar cada uno de forma individual cuando lo necesitemos, de esta forma
no hay que renderizar toda la página cada vez que se cambia un simple selector, sólo la o las
partes necesarias, lo que mejora notablemente el rendimiento y da una experiencia mucho más
fluida al usuario.
\\\\
Tras crear una base de placeholders que diferencien las distintas partes planificadas
de la aplicación, comenzamos con el desarrollo de las primeras características 
partiendo de las primeras historias de usuario.

%Justificar por qué esa biblioteca y hablar un poco de ella
\section{Biblioteca de canvas para React: KonvaJS}
Para empezar a elaborar la aplicación, comenzamos por la funcionalidad más básica necesaria
cargar multimedia en el lienzo del editor, necesitamos una imagen para editar y por
ende se comenzó por crear esta funcionalidad.
\\
Ya se había elaborado una carga de imagen en las demos de la fase de planificación, 
puramente en html y javascript, empleando el elemento HTML Canvas, en estas demos 
se pudo comprobar que para la tarea que queríamos realizar manejar el elemento
canvas desde el DOM\cite{DOM} no era eficiente, ni aprovechaba todas las ventajas 
de React.
El elemento HTML canvas es un mero array de píxeles, no hay manera de diferenciar 
los objetos dibujados en el canvas una vez dibujados, son solo píxeles de otro color,
para hacer esto, tenemos que tener los objetos almacenados en memoria y actualizar el
canvas repintando todos los objetos cada vez que se modifica algo, esto no es una 
tarea fácil, ya que hay que gestionar entre otras cosas la frecuencia de actualización
del canvas (frames per second) y asegurarse de que el código es eficiente, ya que 
no es una tarea trivial en potencia de procesado necesaria.
Por ende y tras investigar las distintas posibilidades se decidió utilizar una 
biblioteca específica para gestionar el elemento canvas HTML para React, de este
modo, podemos aprovechar las características de React y olvidarnos de gestionar 
la frecuencia de actualización del canvas, ya que la biblioteca lo repinta cada vez
que se hace un cambio, además de tener una forma unificada de instanciar los objetos
que están pintados en el canvas, todos los objetos que se pintan son instancias de 
hijos de clases propias de la biblioteca o directamente son instancias de clase de la 
misma.
\\\\
Tras investigar y barajar varias opciones se terminó decidiendo elegir KonvaJS 
\cite{KonvaJS} como biblioteca, primero porque era específica de React
(aunque tiene versiones para Vue.js) 
y segundo, porque contaba con clases y métodos bastante interesantes para una aplicación
como un editor, el resto de bibliotecas estaban mucho más enfocadas a animaciones, sin 
embargo, KonvaJS permite mucha más interacción con el lienzo, que es lo principal en un
editor.
%Como funciona
\section{El Drag\&Drop principal}
%Lo primero que se empezó a hacer despues de poder añadir la imagen
Habiendo importado las dependencias necesarias para KonvaJS\cite{KonvaJS} en el proyecto,
se comenzó a implementar la funcionalidad más básica y necesaria del editor, para editar,
necesitamos tener algo que editar. 
\\
En este punto había que tener en cuenta una de las principales intenciones del proyecto,
se debía poder utilizar el editor sin necesidad de descargar nada o utilizar archivos
guardados en la memoria local, es decir, que había que poder añadir contenido desde 
el portapapeles al editor, aunque también se da la opción de usar contenido local.
\\
Para comenzar, se creó un componente de React que se renderizaría cuando no haya 
nada cargado en el lienzo.

\begin{lstlisting}[caption={Renderizado condicional del componente del DragandDrop}]
  {//If there is no image, show draganddrop input
  ( this.state.image == null ) ?
      <DragandDrop imgLoader={this.imageLoader.bind(this)}/> 
      : 
      <SecondaryDragandDrop 
          imgLoader={this.createNewSecondaryImage.bind(this)}
      />
  }
\end{lstlisting}

Este componente consiste en un área de drop donde el usuario puede añadir el archivo 
haciendo drag desde su PC, se emplea el evento OnDrop para leer el fichero, si es
del formato correcto, se lee y se manda al componente padre (editor) para cargarlo en 
el canvas.
\\
El canvas se carga haciendo uso de la biblioteca KonvaJS \cite{KonvaJS}, desde React 
no tenemos un elemento canvas, tenemos el componente Stage de la biblioteca, así que 
lo cargamos en él, haciendo uso de los métodos de la biblioteca.
\\\\
El canvas se renderiza desde el principio de la aplicación, al principio se pensó 
en hacer un renderizado condicional al igual que con la zona de drop, pero el Stage
(el canvas en el DOM), es el elemento más importante de toda la aplicación
que además es necesario para el funcionamiento base de la propia aplicación 
(no se puede editar nada sin él, es el propio lienzo), así que no tenía mucho sentido
condicionar su renderizado ya que no aportaba nada,
para evitar posibles complicaciones simplemente se renderiza en oculto con ancho y alto
de 0.
\\\\
Cuando se carga una nueva imagen necesitamos calcular qué tamaño tendrá el lienzo, 
cada imagen puede tener un tamaño completamente diferente y queremos que la aplicación
tenga consistencia, es decir, que el usuario sienta que siempre tiene un área de edición
parecida independientemente de la resolución de la imagen.
Para ello debemos tener dos cosas en cuenta, el tamaño de la zona de edición del usuario,
que depende de su propia pantalla y el tamaño original de la imagen.
La correlación de la imagen siempre habrá que mantenerla para que no se deforme la 
imagen base, la calculamos y manteniendo esa correlación ajustamos la imagen al tamaño
disponible dejando algo de espacio por estética (para las imágenes demasiado pequeñas, 
no se efectúa ajuste porque no es necesario).

\begin{lstlisting}[caption={Método para calcular el tamaño del lienzo a partir de una imagen}]
  calculate_resize(correlation, width, height){
        let pixel_margin = 100
        //wide photo
        if( correlation <= 1 ){
            //image too small, dont correct
            width = (width > 400 ? width - pixel_margin : width)
            //
            if( correlation * width < height ){
                height = (correlation * width) - pixel_margin
                width = width - pixel_margin
            } else {
                width = height * (1/correlation) - pixel_margin
                height = height - pixel_margin
            }

        //long photo
        } else {
            //image too small
            height = (height > 400 ? height - pixel_margin : height)
            //
            if( height * (1/correlation) < width ){
                width = height * (1/correlation)
                //height = height
            } else {
                width = width - pixel_margin
                height = correlation * width - pixel_margin
            }
        }

        return {
            width: width,
            height: height
        }
    }
\end{lstlisting}

\newpage
\section{La primera herramienta: Añadir Texto}
Ya que tenemos el mínimo necesario, que es tener una imagen que empezar a modificar, 
comenzamos a implementar la primera funcionalidad, al ser un editor especialmente
enfocado en facilitar la edición de memes, se comenzó por la funcionalidad más básica
para hacer un meme, añadir texto encima de una imagen base.
\\\\
Como hemos comentado, desde React tenemos una abstracción del canvas html llamada Stage,
dentro de este, podemos añadir cualquier objeto nodo de la biblioteca KonvaJS y el texto
ya existe como una clase. Para comenzar el desarrollo añadimos un texto temporalmente
desde el render al que se fue dotando de funcionalidad.
\\
Además del texto, la biblioteca cuenta con otra clase llamada transformer, que sirve 
para mostrar una caja de transformación alrededor de cualquier otro nodo. El problema
de esta clase es que el comportamiento del transformer por defecto consiste en 
reescalar el nodo, en un texto lo que queremos es que las letras se distribuyan
equitativamente por todo el espacio ya que el tamaño se define a parte.
\\\\
Para ello, se ha sobrecargado el comportamiento del evento de transformación del texto.
Ya que sobrecargamos una funcionalidad básica de la clase propia de la biblioteca, se 
decidió crear una clase propia partiendo de la base de la biblioteca, la clase CvText
que encapsula el renderizado del nodo base de Konva con nuestras funcionalidades personalizadas.
Realmente no es una clase, es un componente funcional de React, es decir, es una 
función que devuelve un render, en las nuevas versiones de React se recomienda usar
componentes funcionales ya que simplifican el código y practicamente tienen las mismas
capacidades que las clases gracias a los React-Hooks \cite{React-Hooks}, siempre que 
no se ha necesitado instanciar un componente para llamar métodos asociados a él 
(esta es la principal diferencia, al ser una función no tiene métodos propios) se han
creado componentes funcionales como marcan las buenas prácticas actuales del lenguaje.
\\\\
Por defecto, los métodos cuentan con un tamaño (ancho y alto) y una propiedad de escalado
en X y en Y, el transformador cambia la escala, por ende, cada vez que el nodo se transforma
reseteamos la escala y la aplicamos al tamaño real del nodo. 

\newpage

\begin{lstlisting}[caption={Sobrecarga del reescalado del nodo CvText}]
const scaleReset = () => {
    let text = textRef.current

    text.setAttrs({
        width: text.width() * text.scaleX(),
        scaleX: 1,
        height: text.height() * text.scaleY(),
        scaleY: 1
    })
}
\end{lstlisting}

Para mover el nodo por el lienzo contamos con la propiedad drag de la biblioteca, al ser
una funcionalidad tan básica, venía también preimplementada en la biblioteca.
Más adelante veremos que esta propiedad se hace condicional ya que no siempre queremos
poder mover cada nodo al clickar encima, pero en este punto se añadió directamente para
simular el comportamiento que queríamos que tuviese el nodo.
\\\\
Ahora viene la funcionalidad más importante del texto, poder editar los 
carácteres, para ello, hemos hecho uso del elemento HTML textarea, es mucho más
fácil que crear todo un sistema de edición dentro del lienzo, ya que como hemos comentado
no es más que un array de píxeles. Pero, queremos que el usuario lo perciba como si estuviera
realmente editando el texto de dentro del lienzo de una forma dinámica,
hay un evidente problema con esto ya que la textarea HTML por defecto es una caja opaca
sin estilo alguno, para solucionar este problema se ha creado un método dentro del 
componente CvText que lee el estilo del texto dentro del lienzo y calcula una serie
de reglas CSS que se definen a partir del texto base (el texto de Konva \cite{KonvaJS})
y se aplican a la textarea HTML.

\begin{figure}[!h]
  \centering
    \includegraphics[scale=0.55]{img/textareadiff.png}
  \caption{Ejemplo de cambio de estilo de la textarea por defecto y el aspecto tras las reglas css calculadas a partir del texto del canvas}
\end{figure}

%\newpage
La textarea se genera encima del texto del lienzo y se oculta el texto del canvas,
de este modo, el usuario tiene la sensación deseada, aunque realmente no está tocando
el lienzo, cuando el area textpierde el focus desaparece y se vuelve a renderizar el 
texto en el lienzo con el valor actualizado.

No todos los navegadores tienen exactamente el mismo comportamiento así que hubo que 
introducir algunas reglas extra o excepciones dependiendo del navegador. 
\\\\

\section{El lienzo: Stage}

Ahora que tenemos el primer nodo básico con el funcionamiento deseado, necesitamos que
el usuario sea capaz de añadir y eliminar textos, para ello debemos almacenar los 
nodos que se muestran en el lienzo y repintarlos cada vez que haya algún cambio.
\\\\
Para ello creamos un array que almacena los nodos en el estado del componente, al ser 
un array, React no detecta los cambios automáticamente, así que debemos lanzar manualmente
el repintado del lienzo cuando sea necesario, de este modo tenemos más control sobre 
el renderizado y el rendimiento.
\\\\
Más avanzado el desarrollo, nos encontramos con un problema en este punto. En el desarrollo
de React, se emplea JSX\cite{JSX}, que es una capa de abstracción sobre los métodos de React 
para hacer más intuitivo el desarrollo usando la sintasis de un lenguaje de marcado,
en este caso HTML.
\\\\
Por esto, lo que contiene el array de items no son realmente objetos que se re-renderizan
sino una función propia de React que instancia un objeto camuflado en forma de HTML (JSX\cite{JSX}).
Esto hace que no podamos cambiar las propiedades de los nodos ya renderizados, ya que
en las siguientes iteraciones del render, React detecta que ese nodo ya está instanciado,
el render no hace nada y los cambios no se aplican. 
\\\\
Para solucionar esto hicimos uso de la función React.cloneElement al re-renderizar
un objeto diferente pero que es un clon del anterior (con propiedades modificadas)
que ya está instanciado en caché el re-render sustituye la instancia por la del clon,
con los parámetros cambiados, pero manteniendo los demás que no han sido modificados.

\begin{lstlisting}[caption={Render del array de nodos}]
{ //Renders all items into the canvas
  this.state.itemArray.map(Item => (
      React.cloneElement(
          Item,
          { draggable: (this.state.selectedTool == "SelectorAndText") }
      )
  ))
}
\end{lstlisting}

De este modo los nodos solo pueden ser arrastrados cuando la herramienta correcta está
seleccionada, cuando cambia la herramienta se lanza un re-render del stage y se actualizan
las propiedades de los nodos.

\newpage
\subsection{Los handlers}

Ahora que tenemos una manera de almacenar y modificar los textos, 
el usuario necesita una forma de crearlos y eliminarlos, para ello necesitamos 
detectar los eventos, al igual que con el texto, también necesitamos tratar los eventos
para cada herramienta.
No todas las herramientas tienen los mismos eventos ni se comportan igual ante los eventos,
por eso la forma de capturar cada evento es con un método handler genérico que lleva a un 
switch que redirige el evento al handler correspondiente dependiendo de la herramienta seleccionada.
\\
Los eventos se capturan en el componente Layer, es un componente intermedio propio de KonvaJS\cite{KonvaJS},
que está entre el Stage y los nodos que se renderizan.

\begin{lstlisting}[caption={Enlazado de los handlers al componente layer que captura los eventos}]
<Layer
  ref={this.mainLayerRef}
  onMouseDown={this.handleCanvasMouseDown.bind(this)}
  onMouseMove={this.handleCanvasMouseMove.bind(this)}
  onMouseUp={this.handleCanvasMouseUp.bind(this)}
  onMouseLeave={this.handleCanvasMouseLeave.bind(this)}

  onDblClick={this.handleCanvasDblClick.bind(this)}
>
\end{lstlisting}

Aquí podemos ver un ejemplo de handler genérico del canvas donde se redirige al handler correspondiente

\begin{lstlisting}[caption={Sobrecarga del reescalado del nodo CvText}]
  handleCanvasMouseDown(e){
    switch(this.state.selectedTool){
        case 'SelectorAndText': 
            handleSelectorMouseDown.bind(this)(e)
        break

        case 'FreeLine':
            handleFreeLineMouseDown.bind(this)(e)
        break

        case 'StraightLine':
            handleStraightLineMouseDown.bind(this)(e)
        break

        case 'Rectangle':
            handleRectangleMouseDown.bind(this)(e)
        break

        case 'Ellipse':
            handleEllipseMouseDown.bind(this)(e)
        break
    }
}
\end{lstlisting}

En principio todo este sistema de handlers no se implementó, ya que no era necesario 
porque aún no había más herramientas, en las metodologías ágiles vas implementando según
las funciones que vas necesitando por tanto simplemente se enlazó el handler al evento
de doble click. 
\\\\
El handler crea un nuevo componente de texto (una instancia de nuestro componente propio: CvText),
lo inserta en el array de nodos del editor y lanza un re-render para que se muestre.
Además, en nuestra encapsulación del CvText cuando se crea el nodo por primera vez se
llama automáticamente al método que genera el textarea de edición ya que recién creado
el usuario, muy probablemente querrá editar su contenido.
\\
Para evitar que se lance cuando se clona el elemento, se ha hecho uso de los React Hooks\cite{React-Hooks}
propios de los componentes funcionales.
\\\\
Para el resto de herramientas todo este sistema funciona igual, se detecta un evento, se
redirige al handler de la herramienta correspondiente y se crea un nuevo elemento del tipo
deseado. Dependiendo de cada herramienta el evento con el que se crea el nodo es distinto y el 
comportamiento de los handlers cambia, pero la estructura es la misma. 
\\\\
Hay varios handlers extra que se diferencian de los handlers de las herramientas para los
eventos de teclado como ctrl + v para cargar una imagen o supr para borrar un nodo.
El funcionamiento general es el mismo, pero no se necesita condicionar la herramienta seleccionada
para el borrado simplemente se ejecuta una función que obtiene el nodo seleccionado, lo
destruye de la caché, lo saca del array y re-renderiza el lienzo.
\\\\
Y para la carga de imágenes mediante ctrl + v, se lee el evento, se busca el fichero asociado
en el portapapeles y se envía al método de carga de imagen principal o secundaria dependiendo
de si hay una imagen base cargada.

\newpage
\section{Herramienta: Selector}
Ya que podemos crear el texto necesitamos poder cambiar entre textos, para ello, se creó
la primera herramienta, el selector, que más adelante se fusionó con la creación de texto
por puro pragmatismo, había que estar cambiando de herramienta innecesariamente después
de crear un texto, por eso se decidió unir ambas funcionalidades en una sola herramienta,
para el resto de herramientas no podía ser así ya que coincidian los eventos, el click 
de selección, con por ejemplo, el click que comienza la creación de una línea.
\\\\
La herramienta selectora se encarga de activar y mostrar el componente Transformer en
el nodo que desea el usuario.
Para ello, teníamos que poder diferenciar los nodos del canvas, KonvaJS\cite{KonvaJS}
nos permite encontrar el nodo sobre el que se ha hecho click a partir del evento, 
aprovechando esto, simplemente tomamos el nodo clickado y lo introducimos en el componente
transformer, excepto si lo que se clicka es la imagen base, en cuyo caso vacíamos el transformer
para deseleccionar cualquier nodo.

\section{Cambio de propiedades: BottomToolbar}

%Como funciona el componente bottomtoolbar y cómo va cambiando
%Comentar aquí todo el tema handlers, updates etc y como se sincroniza
%con el cambio de item, es igual para todas las herramientas

Además de cambiar el texto, el usuario debe poder personalizar los elementos que añada al lienzo,
para ello y como teníamos planeado, se añadió una barra de herramientas que irá cambiando
dependiendo del nodo que esté seleccionado.
\\\\
En este punto del desarrollo solo se podían añadir texto, así que se comenzó creando un componente
BottomToolbar para cambiar las propiedades del texto.
Para ello, en primer lugar, necesitamos controladores con los cuales el usuario pueda cambiar 
esas propiedades de forma sencilla, para ello hicimos uso de Material-UI\cite{materialui}, una 
biblioteca de componentes de React con estilo de material design.
\\
\begin{figure}[!h]
  \centering
  \noindent\makebox[\textwidth]{
    \includegraphics[scale=0.3]{img/bottomtoolbar.png}}
  \caption{Aspecto de la BottomToolbar de edición de texto}
\end{figure}

Todos los componentes usados pertenecen a la biblioteca por defecto de Material-UI excepto el
selector de color, que es un proyecto open source independiente hecho a partir y para Material-UI
llamado material-ui-color\cite{material-ui-color}.

\newpage

Teniendo ya los selectores de propiedades creados e insertados en la web, tenemos que hacer
que puedan modificar el texto y viceversa, al seleccionar un texto distinto los selectores 
deben de mostrar los valores actuales del texto en todo momento, es decir, el nodo seleccionado
y la barra de herramientas tienen que estar sincronizados.
\\\\
Aquí nos encontramos con un problema, en React el flujo de datos funciona en una única dirección,
de padres a hijos mediante los props, así que tenemos que encontrar una manera de comunicar y
pasar información entre componentes que no sólo no son padre e hijo, sino que ni siquiera están
anidados y tienen varias capas de abstracción por encima.
\\\\
Para solucionar este problema se optó por aprovechar una de las propiedades de Javascript, 
las funciones también son objetos, objetos a los que se puede ligar un contexto \textbf{this}
dinámico, es decir, que el this no tiene por qué ser la propia función, puede ser su padre como
hemos hecho en este caso.
De este modo, lo que pasamos al componente BottomToolbar son funciones que permiten modificar
las propiedades del nodo seleccionado y son llamadas desde la BottomToolbar pasando el valor 
del selector que está siendo modificado por un evento que genera el usuario.

\begin{figure}[!h]
  \centering
  \noindent\makebox[\textwidth]{
    \includegraphics[scale=0.45]{img/FlujoDatosBottomToolbar.png}}
  \caption{Esquema del flujo de datos entre los nodos del lienzo y la BottomToolbar}
\end{figure}

Como hemos comentado, lo contrario también es cierto, como podemos ver en la anterior figura, 
necesitamos una forma de cambiar el estado de la toolbar al seleccionar un nodo distinto
para que los valores se sincronicen.

\newpage

\begin{lstlisting}[caption={Ejemplo de una función updater de la propiedad de un nodo}]
updateBlur( newValue ){
  let img = this.transformerRef.current.nodes()[0]

  if( img != null){
      //change
      img.blurRadius(newValue)
      //Apply changes
      img.cache()
      //redraw
      img.getStage().batchDraw()
  }
}
\end{lstlisting}

Para cambiar los valores de los selectores de la BottomToolbar al seleccionar un nuevo nodo
tenemos que enviar el valor de las propiedades del nodo a la misma, para esto creamos un 
método que permite cambiar los valores desde el el editor. Cuando se selecciona un nodo
diferente, el handler llama a este método y cambia los valores.
\\\\

\section{Cambio entre herramientas: Toolbar}

Con esta última funcionalidad y esqueleto comenzamos a introducir nuevas herramientas, para 
cambiar de herramientas necesitamos cambiar los métodos anteriormente implementados como se 
comentó en el punto 6.5.1, para que el comportamiento de la aplicación se modifique dependiendo
de la herramienta seleccionada.
\\\\
Como la herramienta seleccionada forma parte del estado del componente principal, cuando hay 
un cambio de herramienta se lanza un re-render del editor y todos sus hijos, esto se ha
implementado de esta forma intencionalmente, ya que al re-renderizar todo nos aseguramos de que
ninguna parte de la aplicación se quedará desincronizada con el estado actual. No nos afecta 
al rendimiento porque es un evento que se lanza de forma puntual, al contrario de otros como
puede ser un evento de movimiento del ratón.
\\\\
No todos los nodos tienen las mismas propiedades, como podemos ver en la figura 6.4,
la BottomToolbar también cambia dependiendo de que nodo se quiera modificar, para esto, 
tuvimos que rediseñar la estructura del componente BottomToolbar para que funcionase como un 
switch que redirija a la toolbar correspondiente dependiendo del estado.
De este modo, el componente BottomToolbar es más bien una interfaz que un componente con una 
funcionalidad propia, todas las diferentes BottomToolbars se han implementado con los mismos
métodos updateToolbar() aunque cambian los parámetros que recogen.

\begin{lstlisting}[caption={Método updateToolbar del componente BottomToolbar general, que actua como interfaz devolviendo el updateToolbar de la toolbar activada}]
  updateToolbar(){
    let bottomToolbar = this.bottomToolbarRef.current
    return bottomToolbar.updateToolbar.bind(bottomToolbar)
  }
\end{lstlisting}

\begin{lstlisting}[caption={Switch del componente BottomToolbar principal que devuelve la BottomToolbar según el estado.}]
  getBottomToolbar(){
    switch(this.state.selectedToolbar){
        case "SelectorAndText":
          return <TextBottomToolbar 
            ref={this.bottomToolbarRef}

            fontSizeUpdater    = {this.fontSizeUpdater}
            fontColorUpdater   = {this.fontColorUpdater}
            strokeColorUpdater = {this.strokeColorUpdater}
            strokeSizeUpdater  = {this.strokeSizeUpdater}
            fontFamilyUpdater  = {this.fontFamilyUpdater}
            alignmentUpdater   = {this.alignmentUpdater}
          />
        break
        case "FreeLine":
          return <LineBottomToolbar 
            ref={this.bottomToolbarRef}

            strokeColorUpdater = {this.strokeColorUpdater}
            strokeSizeUpdater  = {this.strokeSizeUpdater}
            shadowColorUpdater={this.shadowColorUpdater}
            shadowSizeUpdater={this.shadowSizeUpdater}
            dashUpdater={this.dashUpdater}
            />
        break
        case "StraightLine":
          return <LineBottomToolbar 
            ref={this.bottomToolbarRef}

            strokeColorUpdater = {this.strokeColorUpdater}
            strokeSizeUpdater  = {this.strokeSizeUpdater}
            shadowColorUpdater={this.shadowColorUpdater}
            shadowSizeUpdater={this.shadowSizeUpdater}
            dashUpdater={this.dashUpdater}
          />
        break
        case "Rectangle":
        case "Ellipse":
          return <RectBottomToolbar
            ref={this.bottomToolbarRef}

            //we can reuse the rectbottom toolbar but corner radius has no sense
            disableCornerRadius={this.state.selectedToolbar == "Ellipse"}

            strokeColorUpdater = {this.strokeColorUpdater}
            strokeSizeUpdater  = {this.strokeSizeUpdater}
            shadowColorUpdater={this.shadowColorUpdater}
            shadowSizeUpdater={this.shadowSizeUpdater}
            cornerRadiusUpdater={this.cornerRadiusUpdater}
            fillUpdater={this.fillUpdater}
          />
        case "KonvaImage":
          return <ImageBottomToolbar 
            ref={this.bottomToolbarRef}

            strokeColorUpdater = {this.strokeColorUpdater}
            strokeSizeUpdater  = {this.strokeSizeUpdater}
            shadowColorUpdater={this.shadowColorUpdater}
            shadowSizeUpdater={this.shadowSizeUpdater}
            blurValueUpdater={this.blurValueUpdater}
            brightnessValueUpdater={this.brightnessValueUpdater}
            noiseValueUpdater={this.noiseValueUpdater}
            pixelValueUpdater={this.pixelValueUpdater}
            maskValueUpdater={this.maskValueUpdater}
        />
        break
    }
  }
\end{lstlisting}

De este modo se resolvió el problema del comportamiento distinto entre toolbars, ya que cada
una cambia los parámetros que necesita pero englobándolo todo en un solo componente, esto es 
importante para el rendimiento ya que cada vez que se selecciona un nodo al contrario que al
cambiar de herramienta sólo se refresca el componente BottomToolbar y no toda la aplcación,
lo que hace que la experiencia del usuario mejore sustancialmente.
\\\\
Tras la implementación de todo este esqueleto sólo nos faltaba añadir más herramientas, las 
distintas partes del editor ya son capaces de comunicarse de forma eficiente, como hemos
comentado anteriormente, cada herramienta tiene sus propios handlers para definir su comportamiento,
por ejemplo, las líneas se crean clickando y arrastrando y se desactivan el resto de handlers dependiendo
de la herramienta seleccionada.
\\\\
Todos los nodos derivan de clases ya definidas en la biblioteca de Konva\cite{KonvaJS},
se ha creado su bottomtoolbar correspondiente dependiendo de las propiedades que quisiesemos modificar,
sus handlers propios y se han añadido.
\\\\

\newpage
\section{Imágenes superpuestas}

El nodo más destacable de todos los que se pueden añadir son las imágenes superpuestas, no son
una herramienta como tal ya que no hay eventos de ratón más que los propios del selector para
transformar la imagen. Para introducir imágenes podemos usar los mismos eventos que para la imagen
principal, arrastrar para introducir una encima de la principal o añadir desde el portapapeles 
haciendo uso del shortcut. Para implementarlo, hemos reutilizado las funciones que leen la imagen
y adaptado para que se cree un nuevo nodo cuando se termina la carga. 
\\\\
Para el área de drop se creó un elemento HTML oculto que se muestra cuando se detecta que el 
usuario arrastra un archivo, para que tenga un aspecto agradable se jugó con las reglas CSS 
para que tenga una posición absoluta encima de toda la página,
 al ser translúcido da un efecto de oscurecimiento.

\begin{figure}[!h]
  \centering
  \noindent\makebox[\textwidth]{
    \includegraphics[scale=0.4]{img/secondaryadd.png}}
  \caption{Área de drop para imagenes secundarias}
\end{figure}

\newpage
En las imágenes se pueden aplicar filtros desde la BottomToolbar, al contrario que en el resto
de propiedades de otros nodos, para aplicar los filtros había que darle una vuelta de tuerca mas,
puesto que necesitamos guardar la imagen base para aplicar los filtros encima.
Para ello, tenemos que guardar la imagen en caché al crear el nodo y añadirle los filtros con 
valores por defecto, es decir, el filtro está aplicado pero con un valor que no afecta la imagen.

\begin{figure}[!h]
  \centering
  \noindent\makebox[\textwidth]{
    \includegraphics[scale=0.4]{img/ejemploimagenblur.png}}
  \caption{Ejemplo imagen superpuesta con un flitro de blur al máximo.}
\end{figure}

\newpage
\section{Tests}

Como es costumbre en las metodologías ágiles, se ha testeado, en la medida de lo posible,
todo el código que se ha ido desarrollando según se iba añadiendo a la aplicación.
Para ello necesitábamos escoger de qué manera íbamos a testear nuestro código, se barajaron
varias opciones, como el módulo de tests-utils que trae react-dom \cite{react-dom} o jest, 
pero estos módulos no encajan lo suficientemente bien con una aplicación como un editor, 
algo que es muy interactivo y gráfico requería testearlo también de forma gráfica. Por ende
se terminó escogiendo Cypress \cite{cypress} que es una herramienta que simula un navegador de
forma gráfica y genera los eventos, además cuenta con un módulo para React \cite{cypressReact} que permite montar
y testear componentes individualmente, además de comprobar el estado (state) de React, cosa
que no permiten la mayoría de aplicaciones de testeo gráfico, ya que, en el navegador los
componentes no existen como tal, sólo el código HTML que genera React a partir de estos.

\begin{figure}[!h]
  \centering
  \includegraphics[scale=0.6]{img/TestsOK.png}
  \caption{Resultado de la ejecución de los tests de cypress, run-ct}
\end{figure}

Lo ideal es testear el 100\% del código de la aplicación, aunque en la realidad llegar a esto
es bastante complicado, se ha intentado cubrir la mayor parte posible, para ello se han creado
tests para cada historia de usuario que hemos definido, crear el test era requisito para considerar
terminada la historia de usuario y como es lógico el test debía testear la funcionalidad que 
proponía la HU. Hubo una excepción en las primeras historias de usuario que se completaron
no se testearon directamente, ya que el testeo se comenzó algo más tarde para esperar a 
que la aplicación tuviera ciertas funcionalidades básicas completas ya que en el inicio es
muy fácil que ciertas partes cambien drásticamente y así evitar rehacer tests de forma innecesaria,
de todos modos, estas partes fueron testeadas en cuando se empezaron a desarrollar los tests.

\newpage
\section{Despliegue de la aplicación}

Ahora solo nos falta desplegar la aplicación, en nuestro caso se va a desplegar esta primera 
versión utilizando el servicio de Github Pages \cite{GithubPages}.
\\
Como comentamos al inicio, para crear esta aplicación empleamos el paquete create-react-app \cite{create-react-app},
este paquete cuenta con scripts que ayudan en el despliegue de la aplicación y crean una 
versión optimizada del código.
Como era uno de los objetivos del desarrollo, la aplicación ha sido desarrollada pensando
en hacer que, aunque sea una aplicación multimedia, no se requiera de un servidor potente que
soporte mucha carga, es decir, es puro front-end, así que sólo necesitamos que el servidor
envíe código HTML, CSS y los scripts de Javascript, no tiene que procesar nada relativo a las
ediciones que haga el usuario ya que esas operaciones se efectúan en el cliente.
\\
Esto nos permite emplear hostings ligeros para el despliegue, 
%en nuestro caso hemos decidido
%emplear Github Pages por 3 razones, el proyecto ya estaba siendo desarrollado en Github,
%el paquete create-react-app \cite{create-react-app} tiene soporte para Github Pages \cite{GithubPages}
%y por qué no decirlo, es gratuito y funciona bastante bien.
los hostings más usados son Azure, Firebase o AWS aunque son bastante flexibles son hostings 
bastante potentes y de pago, gracias a como está construida la aplicación no necesitamos tanto,
así que vamos a emplear Github Pages por varias razones, el paquete 
create-react-app \cite{create-react-app} tiene soporte para Github Pages \cite{GithubPages}
y por qué no decirlo, es gratuito y funciona bastante bien.
\\
Para el despliegue se han seguido las instrucciones que nos da la documentación de 
create-react-app para hacer deploy en Github Pages \cite{GithubPagesDeploy}.
Es bastante simple, añadimos la configuración de la ruta donde se hará el deploy e instalamos
el paquete npm gh-pages \cite{gh-pages}, que es el que recomienda la documentación oficial.
Y añadimos unos scripts que hacen uso del paquete para el deploy, al ejecutar el script
crea una branch que sirve como source estático de la web.
\\\\
El despliegue puede probarse en el siguiente enlace: \url{https://bytevictor.github.io/memeHub/}